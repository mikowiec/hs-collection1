
\section{Model of the World}


    The challange is to find the aspects of nature which are relevant
    for the result of the models, to find a level on which processing
    time is acceptable and yet the appearance implies an underlaying
    natural regulating system.


% Nedan �r inte helt implementerat �n, men jag har t�nkt g�ra det.. - peter

\subsection{Environmental factors}


    To make the biological complexity managable the model is kept very
    simple, while allowing for a few interesting interactions with the
    environment.
    The environment of the simulation is approximated by four
    factors, temperature, sunlight, substance levels and physical
    structure of the surroundings.

    Some or all of the environmental factors may vary over time. 
    Either because of external changes such as different seasons or
    internal changes caused by the actual simulation (shading, water
    consumption, collision between large branches, ...)
     

\subsubsection*{Temperature}

    An average temperature for the global world is maintained, local
    changes are not taken into account.


\subsubsection*{Sunlight}

    Each point in the world can be queried for the direction towards
    the sun and the intensity of the light. Since these properties
    changes wildly even for shorter periods, they may be approximated
    either by a simple average or a sampling of real values which 
    are adapted for the simulation.

\subsubsection*{Substance levels}

    (currently only an generalized``power'') \\
    Each point in the world, air as well as soil, can be queried for the
    concentration of power available for consumption by the plant.
    This generic ``power'' can of course be split up in a number of
    resources such as water and minerals to allow for more detailed
    simulations, if need arises.


\subsubsection*{Physical structure}

    Each point in the world can be queried whether it is free (in
    the air) or contained by some structure (soil, rocks,
    plants...). To guide the simulation a vector pointing towards
    the closest surface to a given point is also available. 


\subsection{Feedback and limitations}

    As mentioned above, an extremely detailed model of the world is
    not realistic in a realtime simulation, especially modifications
    of the environment caused by the simulated plants is heavily
    simplified. Water levels and light is accounted for in big blocks,
    shading is very limited and coarse, small structures are ignored
    for collision detection, and so on...

    However, even rather simple models give rise to quite natural and
    interesting behaviour of the simulated plants.



