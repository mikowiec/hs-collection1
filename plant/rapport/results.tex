
\section{Comparison of methodologies}


\subsection{Global analysis}


    The  parameterized approach makes use of its global view of the
    tree for a number of reasons. The ability to maintain a strict
    shape and to apply restriction of growth relies on such
    information. Further more, it is necessary to access such
    information when emulating forces of nature. For instance the sun
    emulation is dependent on information of the neighbouring
    branch-environment.

    While possible to some extent, global analysis is generally
    avoided and discouraged with the monadic approach due to the
    complexity and dependencies it causes. Thus some designs and
    behaviours are hard to model, such as advanced resource
    distributions or forcing a certain shape on a plant in the
    presence of resource limitations.


\subsection{Extensibility}

    When attempting to model a plant that is out-of-structure with the
    parametrized system there is need to make changes in some of the
    underlying layer. In the most probable cases it will be enough to
    extend the metalanguage, but severe differences need a change in
    the simulation module. The monadic approach uses a much simpler
    and less hardcoded simulationfunction with most of the logic in
    the actual plant descriptions, allowing extensions to be coded
    completely by the user. However, the relatively primitive language
    may limit the complexity of feasible additions, especially if they
    involve global analysis.

\newpage
\section{Results}

    We successfully simulate a large number of trees within reasonable
    time. They can be simulated in real-time, should the structure not
    be complex, or quasi-realtime when necessary. Further more, it is
    possible to export geometrical representations to be used with
    raytracers for high-quality images.
    




