
\section{Introduction}


    Computer games today drives much of the development in the
    area of computer graphics, both software and especially hardware.
    Producing convincing scenery which mimics the real world as close
    as possible, while still allowing real-time action, is the holy
    grail of the industry.

    Hardware have until recently been the fundamental bottleneck
    behind the visualisation of complex dynamic scenes, thus also
    hindering the development of detailed simulation of events in the
    world. However, the last couple of years the capabilities have
    increased enormously allowing for usage of dynamic and realistic
    visualisations of scenery. Most of these capabilities have been
    used to enhance the main characters of the games, leaving the
    background as dead and boring as before. Much because many games
    take place strictly indoors, or outdoors with limited movement
    possibilities, where a simple background-scene may be sufficient.


\subsection{Plant simulation and visualisation}

    As outdoor scenes becomes increasingly common in gaming scenery,
    so does the importance of representing a visually satisfying
    plant-life. Although some steps have been taken from simple
    texture representations, in order to create a more
    three-dimensional look, plant-life is still often unmixed,
    containing multiple clones of the same model. An outdoor
    environment is seldomly, and then merely scarcely affected by the
    forces and constraints of nature and time, which renders the
    experience only half-living. And the passing of time is, without
    any exception that we are aware of, never taken into account.
    Clearly, it is desirable being able to observe the change of
    seasons and withering of time and limitation of resources.

    Thus we wish to create means to design the plant-life for
    environments supporting such properties. The resulting plants will
    be physically individual and evolve from seed to fully grown trees
    or plants over time. They should respond to changes in available
    resources and strive for their energy source (e.g. the Sun).
    Furthermore, complete environments (e.g.  woods) can be grown in
    realtime during gameplay.


\subsection{Our goal}

    The language we aim to develop should be capable of making
    specifications with enough detail to make convincing models of a
    wide variety of real and imagined plants, both as static
    visualisations and dynamic simulations.  Attributes of the plant,
    both visual appearance and behaviour over time should follow the
    specifications as closely as is desirable. 

    To execute these specifications an interpreter is developed, which
    evolves the plants and visualizes the result. The interpreter is
    also capable of exporting the model to external programs (such as
    ray-tracers), or being used as part of one via a plugin-interface.
    The latter enables 3d-modelers such as Maya or SoftImage to use
    our system to evolve a plant which is then freezed and can be
    finetuned by the modeling-tool.

\subsection{Limitations}

    \label{highlevel_limitations}    
    The focus are set on the graphic result of the simulations.
    That is, we are not attempting to resemble nature in any
    other ways than those affecting the visual appearance of the
    objects. 
