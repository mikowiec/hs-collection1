
\section{Background}


    A number of system and approaces exists for modelling and evolving
    plants and other biological structures, most of them are based on
    Lindenmayer systems, described below. 


\subsection{L-Systems (Lindenmayer Systems)}

\label{lindenmayer}

    Aristid Lindenmayer proposed L-systems as a mathematical
    formalisation of biological development in 1968. They have since
    been successfully used in computer graphics to generate realistic
    models of plants. 

    L-systems build on a very simple idea, successive rewriting of a
    string of symbols where the rules are applied in
    parallel\footnote{A similar system is Chomsky grammars, but the
    rules in these are applied sequentially, which does not reflect
    the parallel workings of nature.}.

    An L-system is described by an alphabet of symbols, a starting
    symbol and a number of rules. Originating from the starting symbol
    a string is built by applying matching rules iteratively.

    Consider the alphabet ${ab}$, starting symbol $a$ and rewrite
    rules ${a\rightarrow ab, b\rightarrow a}$. This successively produces the strings: 
    $a$, $ab$, $aba$, $abaab$, $abaababa$, \ldots 

    Now, of what use is strings of letters to model nature?
    The trick is to interpret the string as a geometrical description,
    on common way, but by no means the only, is to use turtle graphics
    \cite{turtle}.


\begin{figure}[h]

\caption{Simple L-system description and a possible visualisation}
\label{lsys-example}
\end{figure}
    
    Example \ref{lsys-example} is a D0L system, deterministic, context free
    (taking 0 neighbours in account), L-system. In context dependent
    systems (D$N$L) a rule may take a number of neighbouring symbols
    when deciding if it matches or not, however only a single symbol
    is rewritten.
    \todo{mention nondeterministic systems}


\begin{itemize}

    \item \todo{examples of simple L-systems}
    \item \todo{extensions}
    \item \todo{limitations/problems}
    \item \todo{}

\end{itemize}

\subsubsection{Parametric L-system}

\label{param_l_system}

    In \todo{cite} and \todo{cite} a number of parameters and
    conditions on these are also included in the rules, allowing for
    simpler descriptions of what would otherwise be quite
    incomprehensible (parameters can be simulated by extending the
    alphabet). \todo{example}


\subsection{Further extensions}

    Originating from this very simple idea, the L-systems have been
    extended and improved to support several interesting features such
    as signaling and bug-attacks. They do however fast tend to get
    complicated to grasp as they grow in complexity. One idea early on
    was simply to write a shell around this framework, to make it
    easier and more intuitive to use. It was however concluded that
    this approach might have caused unnecesarry limitations. Instead
    it was agreed to begin at a higher level and perhaps later export
    to L-systems syntax.



