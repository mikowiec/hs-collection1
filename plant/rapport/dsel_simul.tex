




\subsection{Overview}


    This approach is essentially just an L-system with a
    rather big state as parameters and a dynamic number of
    structure parts.  However, focus is moved from the
    explicit and complete state changes towards incremental
    changes and especially their triggers and high-level
    effects. It is possible to describe events abstractly
    and capture repeatedly occurring patterns in higher
    order functions.

    In addition to the actual programs, a simulation module
    and a visualization module completes the design. These
    two are described in detail in the implementation
    section (\ref{monadic_impl}). The division in two,
    rather independent, passes, simulation and
    visualization, provides two interesting features:

    When simulating a new world, to a given state without
    caring about visualization (until finished), the
    first pass may be run much more often than the
    second, trading accuracy for speed (the visualization
    is be used to calculate feedback originating from
    spacial orientation, which is not directly known in the
    structure tree, more on this in section
    \ref{monadic_feedback}).

    When visualizing a "constant" world (time changing
    slowly, or not at all), the second pass can be used to
    apply small mutations to the model, to give the
    impression of windy circumstances (rattling leaves,
    bending branches) and also to add detail and
    irregularities. 

    The second pass may also be used to interpolate between
    discrete states, providing smaller steps, and animation,
    cheaply.



\subsection{Basics}

    A plant consists of a number of structure parts
    organized in an tree (or a DAG, given the shared
    representation optimization described below). Each node
    have a state and a program associated with it. 

    The state can be seen as the parameters of a parametric
    L-System (see \ref{param_l_system}) and is essentially
    divided in two parts, one for parameters which are
    needed for the simulation and visualization and one for
    the private, internal state of each node.

    The type of the internal state can be different for
    different node types and decides which operations can be
    used in the program. A number of standard sets of
    operations are provided which capture most of the common
    needs for a plant, such as growth, mass (gravity) and
    energy distribution.





\subsection{Overview}



\subsection{Simulation}



\subsection{Visualization}


\subsection{Operations}
\subsubsection{Growth}



\subsubsection{Resources}

\subsubsection{Message propagation}


\subsubsection{Local state}


\subsection{Forests}


\subsection{Conversations from L-systems}

\newpage
\section{Monadic Implementation}

    \label{dsel_impl}
    \label{monadic_impl}

    \todo{impl. strategy}

\subsection{Overview}

    \todo{describe the various parts}

\subsection{Efficiency considerations}


\subsection{Programs}

\subsection{Simulation}


\subsection{Combining Monadic State}

